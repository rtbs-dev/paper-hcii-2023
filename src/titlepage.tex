\title{You Can Tag It, Your Machine Can Help}
\subtitle{Visual Information Seeking and Artificial Intelligence Augmentation in Technical Language Processing}
%
\titlerunning{You Can Tag It, Your Machine Can Help}
% If the paper title is too long for the running head, you can set
% an abbreviated paper title here
%
\author{First Author\inst{1}\orcidID{0000-1111-2222-3333} \and
Second Author\inst{2,3}\orcidID{1111-2222-3333-4444} \and
Third Author\inst{3}\orcidID{2222--3333-4444-5555}}
%
\authorrunning{F. Author et al.}
% First names are abbreviated in the running head.
% If there are more than two authors, 'et al.' is used.
%
\institute{NIST
\email{thurston.sexton@nist.gov}\\
\url{TBD} 
}
%
\maketitle              % typeset the header of the contribution
%
\begin{abstract}
For domain experts to benefit from advances in Natural Language Processing (NLP), a new approach toward annotating and structuring technical text is required, driven by community development of new tools.
We propose a user-centered approach to data labeling that aids in achieving what we term Technical Language Processing (TLP).
We suggest a mapping between visual-information-seeking tasks and the problem of understanding, organizing, and annotating technical text.
Further, we provide an overview of ongoing efforts to this end.
Engagement from HCI and Human Factors community should be central to achieving what we see as fundamentally a form of Artificial Intelligence Augmentation (AIA), so that experts not only benefit from burgeoning algorithmic systems, but are able to trust the systems and the decisions they support.


\keywords{First keyword  \and Second keyword \and Another keyword.}
\end{abstract}